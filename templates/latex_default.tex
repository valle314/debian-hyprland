\documentclass{article}
\usepackage{amsthm}
\usepackage{ngerman}
\usepackage{enumitem}   
\usepackage{amssymb}
\usepackage{mathtools}
\usepackage{theoremref}
\usepackage{mathrsfs}
\usepackage{faktor}

\usepackage{hyperref}
\hypersetup{
    colorlinks=true,
    linkcolor=black,
    filecolor=black,      
    urlcolor=black,
    citecolor = black
}


\usepackage{graphicx}
% \graphicspath{ {./images/} }

\usepackage{geometry}
\geometry{textwidth = 14cm}

\numberwithin{equation}{section}
\newtheorem{theorem}[equation]{Theorem}

\newtheorem{kor}[equation]{Korollar}
\newtheorem{lemma}[equation]{Lemma}
\newtheorem{satz}[equation]{Satz}
\newtheorem{ds}[equation]{Definition und Satz}
\newtheorem{prop}[equation]{Proposition}

\theoremstyle{definition}
\newtheorem{defi}[equation]{Definition}
\newtheorem{bem}[equation]{Bemerkung}
\newtheorem*{notation}{Notation}
\newtheorem*{bew}{Beweis}
\newtheorem*{bsp}{Beispiel}

\makeatletter
\newcommand{\bigp@rp}[2]{%
  \vcenter{
    \m@th\hbox{\scalebox{\ifx#1\displaystyle2.1\else1.5\fi}{$#1\perp$}}
  }%
}
\newcommand{\bigperp}{%
  \mathop{\mathpalette\bigp@rp\relax}%
  \displaylimits
}

\newcommand{\stacka}[1]{\stackrel{\mathmakebox[\widthof{=}]{\mathrm{#1}}}{=}}
\newcommand{\stackb}[2]{\stackrel{\mathmakebox[\widthof{=}]{\mathrm{#2}}}{#1}}
\makeatother


\begin{document}


%Literatur
\bibliographystyle{unsrt}


%Inhaltsverzeichnis
\tableofcontents
\newpage

%--------------------------------------------------------------------------------------------------------------------------------------------------------------------------------------------------------


\section{First section}
First section
\subsection{sub}
Subsection


%--------------------------------------------------------------------------------------------------------------------------------------------------------------------------------------------------------

\end{document}
